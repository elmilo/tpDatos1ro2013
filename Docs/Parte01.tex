\section{Definiciones básicas}

Para diagramar el trabajo primero estableceremos las siguientes definiciones básicas:
\begin{itemize}

%\item Consulta (q): es una ni idea que poner

\item Documento: conjunto de datos en formato estándar de texto (Unicode).

\item Carpeta a indexar: es la que se le pasa al programa creado para que procese todos sus documentos, y permita hacer consultas sobre su contenido

\item Colección: conjunto de todos los documentos que aparecen en la carpeta a indexar.

\end{itemize}


%\subsection{Término (\textit{\textnormal{T}})}
\subsection{Término (\textnormal{\itshape T})}

 
Se considera término \textit{T} a todo conjunto de caracteres alfanumérico sumados al apóstrofo (o comilla simple: ' ). Cada término es, o bien, una cantidad numérica desde 1 (una) hasta una cantidad no definida de cifras, o bien palabras tanto simples como compuestas. Se desarrolla sobre este tema en la Sección \ref{sec:parser}.


\subsection{docID (\textnormal{\itshape D})}
\noindent (\textit{document identification} = identificador de documento)

Este es un número asignado unívocamente a cada documento de la carpeta a indexar. Para hacer este proceso, se listan recursivamente todos los documentos (que se suponen \textit{indexables}, es decir, de texto plano) y se los ordena alfabéticamente. Este orden otorga el docID de cada documento, que comienza desde el 1 (uno).


\subsection{d-gap}
\noindent(\textit{document-gap} = distancia entre documentos)

Considerando una lista de docIDs en orden ascendente, se puede almacenar la misma con el primer elemento seguido de las diferencias entre los elementos siguientes, los \textit{d-gaps}. Se utiliza la definición dada en \citet[p.~115]{WittenMoffatBell99}.

Por ejemplo $\langle 1, 2, 4 , 5 , 8 \rangle$ se puede transformar en $\langle 1, 1, 2 , 1 , 3 \rangle$.



%\citeauthor[p.~115]{WittenMoffatBell99}.

%\subsection{Frecuencia absoluta (\textnormal{\itshape F \texorpdfstring{\textsubscript{F}})}}
\subsection{Frecuencia absoluta (\texorpdfstring{$F_{T}$}{FF})}

Cantidad de veces que aparece un término \textit{T} en toda la colección.


%\subsection{Frecuencia relativa (\textnormal{\itshape f \texorpdfstring\textsubscript{D,T}})}
\subsection{Frecuencia relativa (\texorpdfstring{$F_{D,T}$}{FFD})}

Cantidad de veces que aparece un término \textit{T} en un documento \textit{D}.

\subsection{Lista de posiciones}

También llamada lista de ocurrencias. Es aquella que almacena las posiciones en un documento \textit{D} en donde aparece el término \textit{T}. Una posición en un documento corresponde a la cantidad de términos hasta el término en consideración, empezando por el 1.